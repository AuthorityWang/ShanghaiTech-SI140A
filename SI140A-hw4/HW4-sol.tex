\documentclass{article}

\usepackage{fancyhdr}
\usepackage{extramarks}
\usepackage{amsmath}
\usepackage{amsthm}
\usepackage{amsfonts}
\usepackage{tikz}
\usepackage[plain]{algorithm}
\usepackage{algpseudocode}
\usepackage{enumerate}
\usepackage{tikz}

\usetikzlibrary{automata,positioning}

%
% Basic Document Settings
%  

\topmargin=-0.45in
\evensidemargin=0in
\oddsidemargin=0in
\textwidth=6.5in
\textheight=9.0in
\headsep=0.25in

\linespread{1.1}

\pagestyle{fancy}
\lhead{\hmwkAuthorName}
\chead{\hmwkClass : \hmwkTitle}
\rhead{\firstxmark}
\lfoot{\lastxmark}
\cfoot{\thepage}

\renewcommand\headrulewidth{0.4pt}
\renewcommand\footrulewidth{0.4pt}

\setlength\parindent{0pt}

%
% Create Problem Sections
%

\newcommand{\enterProblemHeader}[1]{
    \nobreak\extramarks{}{Problem \arabic{#1} continued on next page\ldots}\nobreak{}
    \nobreak\extramarks{Problem \arabic{#1} (continued)}{Problem \arabic{#1} continued on next page\ldots}\nobreak{}
}

\newcommand{\exitProblemHeader}[1]{
    \nobreak\extramarks{Problem \arabic{#1} (continued)}{Problem \arabic{#1} continued on next page\ldots}\nobreak{}
    \stepcounter{#1}
    \nobreak\extramarks{Problem \arabic{#1}}{}\nobreak{}
}

\newcommand*\circled[1]{\tikz[baseline=(char.base)]{
		\node[shape=circle,draw,inner sep=2pt] (char) {#1};}}


\setcounter{secnumdepth}{0}
\newcounter{partCounter}
\newcounter{homeworkProblemCounter}
\setcounter{homeworkProblemCounter}{1}
\nobreak\extramarks{Problem \arabic{homeworkProblemCounter}}{}\nobreak{}

%
% Homework Problem Environment
%
% This environment takes an optional argument. When given, it will adjust the
% problem counter. This is useful for when the problems given for your
% assignment aren't sequential. See the last 3 problems of this template for an
% example.
%

\newenvironment{homeworkProblem}[1][-1]{
    \ifnum#1>0
        \setcounter{homeworkProblemCounter}{#1}
    \fi
    \section{Problem \arabic{homeworkProblemCounter}}
    \setcounter{partCounter}{1}
    \enterProblemHeader{homeworkProblemCounter}
}{
    \exitProblemHeader{homeworkProblemCounter}
}

%
% Homework Details
%   - Title
%   - Class
%   - Due date
%   - Name
%   - Student ID

\newcommand{\hmwkTitle}{Homework\ \#04}
\newcommand{\hmwkClass}{Probability \& Statistics for EECS}
\newcommand{\hmwkDueDate}{Mar 12, 2023}
\newcommand{\hmwkAuthorName}{Penghao Wang}
\newcommand{\hmwkAuthorID}{2021533138}


%
% Title Page
%

\title{
    \vspace{2in}
    \textmd{\textbf{\hmwkClass:\\  \hmwkTitle}}\\
    \normalsize\vspace{0.1in}\small{Due\ on\ \hmwkDueDate\ at 23:59}\\
	\vspace{4in}
}

\author{
	Name: \textbf{\hmwkAuthorName} \\
	Student ID: \hmwkAuthorID}
\date{}

\renewcommand{\part}[1]{\textbf{\large Part \Alph{partCounter}}\stepcounter{partCounter}\\}

%
% Various Helper Commands
%

% Useful for algorithms
\newcommand{\alg}[1]{\textsc{\bfseries \footnotesize #1}}
% For derivatives
\newcommand{\deriv}[1]{\frac{\mathrm{d}}{\mathrm{d}x} (#1)}
% For partial derivatives
\newcommand{\pderiv}[2]{\frac{\partial}{\partial #1} (#2)}
% Integral dx
\newcommand{\dx}{\mathrm{d}x}
% Alias for the Solution section header
\newcommand{\solution}{\textbf{\large Solution}}
% Probability commands: Expectation, Variance, Covariance, Bias
\newcommand{\E}{\mathrm{E}}
\newcommand{\Var}{\mathrm{Var}}
\newcommand{\Cov}{\mathrm{Cov}}
\newcommand{\Bias}{\mathrm{Bias}}

\begin{document}

\maketitle

\pagebreak

\begin{homeworkProblem}[1]

\begin{enumerate}
    \item 
    Denote that event A is that the strategy of always switching succeeds, and denote that $B_i, i \in {1, 2, 3}$ is the event that the car is behind door $i$. Then we have \\$$\begin{aligned}
        P(A) &= P(A | B_1) P(B_1) + P(A | B_2) P(B_2) + P(A | B_3) P(B_3)\\
             &= 0 * \dfrac{1}{3} + 1 * \dfrac{1}{3} + 1 * \dfrac{1}{3}\\
             &= \dfrac{2}{3}
    \end{aligned}$$
    So we get the unconditional probability that the strategy of always switching succeeds. \\
    \item 
    Denote that event $C_j$ is that Monty opens the door $j$. Then we need to calculate $P(A | C_2)$, as we assume that we always choose the door 1 first, and event A is that strategy of always switching succeeds, then $P(A | C_2) = P(B_3 | C_2)$ as we door 2 is opened and after switch, we win. \\
    $$\begin{aligned}
        P(B_3 | C_2) &= \dfrac{P(C_2 | B_3)P(B_3)}{P(C_2)}\\
             &= \dfrac{P(C_2 | B_3)P(B_3)}{P(C_2 | B_1)P(B_1) + P(C_2 | B_2)P(B_2) + P(C_2 | B_3)P(B_3)} \\
             &= \dfrac{1 * \dfrac{1}{3}}{p * \dfrac{1}{3} + 0 * \dfrac{1}{3} + 1 * \dfrac{1}{3}} \\
             &= \dfrac{1}{1 + p}
    \end{aligned}$$
    \item
    The same as the question (b). 
    Then we need to calculate $P(A | C_3)$, as we assume that we always choose the door 1 first, and event A is that strategy of always switching succeeds, then $P(A | C_3) = P(B_2 | C_3)$ as we door 3 is opened and after switch, we win. \\
    $$\begin{aligned}
        P(B_2 | C_3) &= \dfrac{P(C_3 | B_2)P(B_3)}{P(C_3)}\\
             &= \dfrac{P(C_3 | B_2)P(B_2)}{P(C_3 | B_1)P(B_1) + P(C_3 | B_2)P(B_2) + P(C_3 | B_3)P(B_3)} \\
             &= \dfrac{1 * \dfrac{1}{3}}{(1-p) * \dfrac{1}{3} + 1 * \dfrac{1}{3} + 0 * \dfrac{1}{3}} \\
             &= \dfrac{1}{2 - p}
    \end{aligned}$$
\end{enumerate}

\end{homeworkProblem}

\newpage

\begin{homeworkProblem}[2]

\begin{enumerate}
    \item Consider the PMF be $\dfrac{a}{n}$, where a is a constant, then we calculate the sum of PMF, which should equal 1. However, $\sum_{i = 1} ^ \infty \dfrac{a}{n} = a\sum_{i = 1} ^ \infty \dfrac{1}{n}$ is a series that do not converge, so there do not exist a constant $a$ that makes the sum of PMF is not equal to 1. So there is not a discrete distribution that makes the value of the PMF at n is proportional to $\dfrac{1}{n}$. \\
    \item Consider the PMF be $\dfrac{b}{n ^ 2}$, where b is a constatn, then we calculate the sum of PMF, which should equal 1. \\
        $$\sum_{i = 1} ^ n \dfrac{b}{n^2} = b\sum_{i = 1} ^ n \dfrac{1}{n^2} = b \dfrac{\pi ^ 2}{6} = \dfrac{b \pi ^ 2}{6} = 1. $$ \\
        $$b = \dfrac{6}{\pi ^ 2}. $$\\
        Then find the satisfied discrete distribution with PMF of $\dfrac{6}{\pi ^ 2 n ^ 2}$ such that the value of the PMF at n is proportional to $\dfrac{1}{n ^ 2}$. 
\end{enumerate}

\end{homeworkProblem}

\newpage

\begin{homeworkProblem}[3]

\begin{enumerate}
    \item X and Y have the same distribution, as X is a random day of the week, so the probability of X = any of the 7 days are equal and is $\dfrac{1}{7}$. Then as Y is the day after X, so the probability of Y = any of the 7 days are also equal and is also $\dfrac{1}{7}$. So X and Y have the same distribution. \\
    \item To calculate the $P(X \textless Y)$, we can calculate the $P(X \geq Y)$ first, there is only case, that is X = 7, Y = 1, the cases' probability is $\dfrac{1}{7}$ as there are 7 days in total. Then the probability $P(X < Y) = 1 - \dfrac{1}{7} = \dfrac{6}{7}$. 
\end{enumerate}

\end{homeworkProblem}

\newpage

\begin{homeworkProblem}[4]

\begin{enumerate}
    \item Firstly we find the range of X, that is $[0, n]$ as there are n flips in total. Define that k is the number of times it lands Heads, then we calculate the PMF of X is that: \\
        $$P(X = k) = \dfrac{1}{2} \begin{pmatrix} n \\ k \end{pmatrix} p_1 ^ k (1 - p_1) ^ {n - k} + \dfrac{1}{2} \begin{pmatrix} n \\ k \end{pmatrix} p_2 ^ k (1 - p_2) ^ {n - k}, \text{where k} \in [0, n].$$ \\
    \item If $p_1 = p_2$, then $$P(X = k) = \begin{pmatrix} n \\ k \end{pmatrix} p_1 ^ k (1 - p_1) ^ {n - k}. $$
        Which is the binomial distribution. \\
    \item As the 2 coins has different probability of landing Heads, then we need to consider the case of choose coin 1 and coin 2, as each time, the coin is fixed and not choose from the 2 coins, so each flip is not independent. So the distribution of X is not binomial distribution. \\
\end{enumerate}

\end{homeworkProblem}

\newpage

\begin{homeworkProblem}[5]

\begin{enumerate}
    \item Consider conditioning on X, then we have that 
    $$\begin{aligned} P(X \oplus Y = 1 | X = 0) 
        &= P(Y = 1)P(X = 0)\\
        &= \dfrac{1 - p}{2}.\end{aligned}$$ 
    $$\begin{aligned} P(X \oplus Y = 1 | X = 1) 
        &= P(Y = 0)P(X = 1)\\
        &= \dfrac{p}{2}.\end{aligned}$$
    Then we have that $P(X \oplus Y = 0) = \dfrac{1}{2} * p + \dfrac{1}{2} * (1 - p) = \dfrac{1}{2}$, then we have $P(X \oplus Y = 1) = \dfrac{1}{2} * p + \dfrac{1}{2} * (1 - p) = \dfrac{1}{2}$. So the distribution of $X \oplus Y$ is Bern($\dfrac{1}{2}$). \\
    \item 
    \begin{enumerate}
        \item X:
        From question (a) we can see that $X \oplus Y$ is independent with X as conditioning on X do not have connection with X. 
        \item Y:
        \begin{enumerate}
            \item If p = $\dfrac{1}{2}$, then we conditioning on Y, we get that with $m \in {0, 1}, n \in {0, 1}$
                $$\begin{aligned}P(X \oplus Y = m, Y = n) &= \dfrac{1}{2}P(X \oplus n = m | m = 0) + \dfrac{1}{2}P(X \oplus n = m | m = 1)\\
                 &= \dfrac{1}{2}P(X \oplus n = 0) + \dfrac{1}{2}P(X \oplus n = 1) \\
                 &= \dfrac{1}{2} * \dfrac{1}{2} * \dfrac{1}{2} + \dfrac{1}{2} * \dfrac{1}{2} * \dfrac{1}{2} \\
                 &= \dfrac{1}{4}\end{aligned}$$ 
                $$P(X \oplus Y = m)P(Y = n) = \dfrac{1}{2} * \dfrac{1}{2} = \dfrac{1}{4}$$
                $$P(X \oplus Y = m, Y = n) = P(X \oplus Y = m)P(Y = n)$$\\
                then the distribution of $X \oplus Y$ is independent with Y. 
            \item If p $\neq \dfrac{1}{2}$, then we conditioning on Y, we get that, for example: 
                $$P(X \oplus Y = 1, Y = 1) = \dfrac{1}{2}P(X \oplus 1 = 1) = \dfrac{1}{2}P(X = 0) = \dfrac{1 - p}{2}$$ 
                $$P(X \oplus Y = 1)P(Y = 1) = \dfrac{1}{2} * \dfrac{1}{2} = \dfrac{1}{4} \neq \dfrac{1 - p}{2}$$\\
                then the distribution of $X \oplus Y$ is not independent with Y. 
        \end{enumerate}
    \end{enumerate}
    \item 
    \begin{enumerate}
        \item Proof of $Y_j \sim Bern(\dfrac{1}{2})$:\\
            When $j = 1$, then $Y_j = X_1$, then we have $P(Y_j = 1) = P(X_1 = 1) = \dfrac{1}{2}$, so we get $Y_j \sim Bern(1/2)$, with $j = 1$\\
            When $j = 2$, then $Y_j = X_1 \oplus X_2$, then we have $P(Y_j = 1) = P(X_1 \oplus X_2 = 1) = \dfrac{1}{4} + \dfrac{1}{4} = \dfrac{1}{2}$, so $Y_j \sim Bern(1/2)$. \\
            Now assume that $j = k$, $Y_j \sim Bern(1/2)$, then we consider when $j' = k + 1$, then $j' = k + 1$. Then $Y_j' = Y_j \oplus X_{k+1}$. Then $P(Y_j' = 1) = P(Y_j \oplus X_{k+1} = 1) = \dfrac{1}{2}P(Y_j \oplus X_{k+1}=1 | X_{k+1} = 0) + \dfrac{1}{2}P(Y_j \oplus X_{k+1}=1 | X_{k+1} = 1) = \dfrac{1}{2}*\dfrac{1}{2} + \dfrac{1}{2}*\dfrac{1}{2} = \dfrac{1}{2}$. \\
            The same for $P(Y_j = 0)$, so with mathematical induction we have $Y_J \sim Bern(1/2)$. \\
        \item Pairwise independent:\\
            Consider 2 subsets J, K of \{1, 2, ..., n\}, then we have that $J = A \oplus B, K = A \oplus C$, $J\cup K$ can parted into $J\cap K$, $J\cap K^c$, $J^c \cap K$. \\
            Assume that $J \cap K$ is nonempty, then for $y \in {0, 1}, z \in {0, 1}$, we have that $$\begin{aligned}P(Y_j = y, Y_k = z) &= \dfrac{1}{2} P(A \oplus B = y, A \oplus C = z | A = 1) + \dfrac{1}{2} P(A \oplus B = y, A \oplus C = z | A = 0)\\
            &= \dfrac{1}{2} P(1 \oplus B = y) P(1 \oplus C = z) + \dfrac{1}{2} P(0 \oplus B = y)P(0 \oplus C = z)\\
            &= \dfrac{1}{8} + \dfrac{1}{8}\\
            &= \dfrac{1}{4}\\
            &= P(Y_J = y)P(Y_K = z)\end{aligned}$$
            As A, B, C are independent, A, B, C, $Y_J$, $Y_K$ $\sim Bern(1/2)$, we have $Y_J$ and $Y_K$ are independent. \\
        \item Not independent:\\
            Give an example that set A and set B are 2 subsets of \{1, 2, ..., n\}, and satisfy that $A \cap B = \emptyset$, then $A \cup B$ is also a subset of \{1, 2, ..., n\}. Then $P(Y_A = 0, Y_B = 0, Y_{A \cup B} = 0) = 0$, while $P(Y_A = 0)P(Y_B = 0)P(Y_{A\cap B} = 0) = \dfrac{1}{2} * \dfrac{1}{2} * \dfrac{1}{2} = \dfrac{1}{8}$, so not independent. 
    \end{enumerate}
\end{enumerate}

\end{homeworkProblem}

\end{document}
