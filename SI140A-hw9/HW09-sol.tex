\documentclass{article}

\usepackage{fancyhdr}
\usepackage{extramarks}
\usepackage{amsmath}
\usepackage{amsthm}
\usepackage{amsfonts}
\usepackage{tikz}
\usepackage[plain]{algorithm}
\usepackage{algpseudocode}
\usepackage{enumerate}
\usepackage{tikz}

\usetikzlibrary{automata,positioning}

%
% Basic Document Settings
%  

\topmargin=-0.45in
\evensidemargin=0in
\oddsidemargin=0in
\textwidth=6.5in
\textheight=9.0in
\headsep=0.25in

\linespread{1.1}

\pagestyle{fancy}
\lhead{\hmwkAuthorName}
\chead{\hmwkClass : \hmwkTitle}
\rhead{\firstxmark}
\lfoot{\lastxmark}
\cfoot{\thepage}

\renewcommand\headrulewidth{0.4pt}
\renewcommand\footrulewidth{0.4pt}

\setlength\parindent{0pt}

%
% Create Problem Sections
%

\newcommand{\enterProblemHeader}[1]{
    \nobreak\extramarks{}{Problem \arabic{#1} continued on next page\ldots}\nobreak{}
    \nobreak\extramarks{Problem \arabic{#1} (continued)}{Problem \arabic{#1} continued on next page\ldots}\nobreak{}
}

\newcommand{\exitProblemHeader}[1]{
    \nobreak\extramarks{Problem \arabic{#1} (continued)}{Problem \arabic{#1} continued on next page\ldots}\nobreak{}
    \stepcounter{#1}
    \nobreak\extramarks{Problem \arabic{#1}}{}\nobreak{}
}

\newcommand*\circled[1]{\tikz[baseline=(char.base)]{
		\node[shape=circle,draw,inner sep=2pt] (char) {#1};}}


\setcounter{secnumdepth}{0}
\newcounter{partCounter}
\newcounter{homeworkProblemCounter}
\setcounter{homeworkProblemCounter}{1}
\nobreak\extramarks{Problem \arabic{homeworkProblemCounter}}{}\nobreak{}

%
% Homework Problem Environment
%
% This environment takes an optional argument. When given, it will adjust the
% problem counter. This is useful for when the problems given for your
% assignment aren't sequential. See the last 3 problems of this template for an
% example.
%

\newenvironment{homeworkProblem}[1][-1]{
    \ifnum#1>0
        \setcounter{homeworkProblemCounter}{#1}
    \fi
    \section{Problem \arabic{homeworkProblemCounter}}
    \setcounter{partCounter}{1}
    \enterProblemHeader{homeworkProblemCounter}
}{
    \exitProblemHeader{homeworkProblemCounter}
}

%
% Homework Details
%   - Title
%   - Class
%   - Due date
%   - Name
%   - Student ID

\newcommand{\hmwkTitle}{Homework\ \#09}
\newcommand{\hmwkClass}{Probability \& Statistics for EECS}
\newcommand{\hmwkDueDate}{Apr 16, 2023}
\newcommand{\hmwkAuthorName}{Wang Penghao}
\newcommand{\hmwkAuthorID}{2021533138}


%
% Title Page
%

\title{
    \vspace{2in}
    \textmd{\textbf{\hmwkClass:\\  \hmwkTitle}}\\
    \normalsize\vspace{0.1in}\small{Due\ on\ \hmwkDueDate\ at 23:59}\\
	\vspace{4in}
}

\author{
	Name: \textbf{\hmwkAuthorName} \\
	Student ID: \hmwkAuthorID}
\date{}

\renewcommand{\part}[1]{\textbf{\large Part \Alph{partCounter}}\stepcounter{partCounter}\\}

%
% Various Helper Commands
%

% Useful for algorithms
\newcommand{\alg}[1]{\textsc{\bfseries \footnotesize #1}}
% For derivatives
\newcommand{\deriv}[1]{\frac{\mathrm{d}}{\mathrm{d}x} (#1)}
% For partial derivatives
\newcommand{\pderiv}[2]{\frac{\partial}{\partial #1} (#2)}
% Integral dx
\newcommand{\dx}{\mathrm{d}x}
% Alias for the Solution section header
\newcommand{\solution}{\textbf{\large Solution}}
% Probability commands: Expectation, Variance, Covariance, Bias
\newcommand{\E}{\mathrm{E}}
\newcommand{\Var}{\mathrm{Var}}
\newcommand{\Cov}{\mathrm{Cov}}
\newcommand{\Bias}{\mathrm{Bias}}

\begin{document}

\maketitle

\pagebreak

\begin{homeworkProblem}[1]

\begin{enumerate}[(a)]
    \item As for the case X, Y are discrete, we have that use the definition of conditioning probability
        $$P(Y = y | X = x) = \dfrac{P(X = x, Y = Y)}{P(X = x)}$$
        $$P(X = x | Y = y) = \dfrac{P(X = x, Y = Y)}{P(Y = y)}$$
        So that we have that $$P(Y = y | X = x) = \dfrac{P(X = x, Y = Y)}{P(X = x)} = \dfrac{P(X = x | Y = y)P(Y = y)}{P(X = x)}$$
    \item As for the case X, Y are continuous, we have that use the definition of conditioning probability
        $$f_{Y|X}(y | x) = \dfrac{f_{Y, X}(y, x)}{f_X(x)}$$
        $$f_{X|Y}(x | y) = \dfrac{f_{X, Y}(x, y)}{f_Y(y)}$$
        So that we have that $$f_{Y | X}(y | x) = \dfrac{f_{Y, X}(y, x)}{f_X(x)} = \dfrac{f_{X | Y}(x | y)f_Y(y)}{f_X(x)}$$
    \item As for the case X is discrete, Y is continuous, we have that use the definition of conditioning probability
        Denote that $\varepsilon$ is a small number, then we have that
        $$P[Y \in (y - \varepsilon, y + \varepsilon) | X = x] = \dfrac{P[X = x | Y \in (y - \varepsilon, y + \varepsilon)]P[Y \in (y - \varepsilon, y + \varepsilon)]}{P(X = x)}$$
        $$\begin{aligned}
            f_Y(y | X = x) &= \lim_{\varepsilon \to 0} \dfrac{P[Y \in (y - \varepsilon, y + \varepsilon) | X = x]}{2\varepsilon} \\
                &= \lim_{\varepsilon \to 0} \dfrac{P[X = x | Y \in (y - \varepsilon, y + \varepsilon)]\dfrac{P[Y \in (y - \varepsilon, y + \varepsilon)]}{2 \varepsilon}}{P(X = x)} \\
                &= \dfrac{P(X = x | Y = y)f_Y(y)}{P(X = x)}
        \end{aligned}$$
        So that we have that $$f_Y(y | X = x) = \dfrac{P(X = x | Y = y)f_Y(y)}{P(X = x)}$$
    \item As for the case X is continuous, Y is discrete, we have that use the definition of conditioning probability
        Denote that $\varepsilon$ is a small number, then we have that
        $$\begin{aligned} \lim_{\varepsilon \to 0} P(Y = y | X \in (x - \varepsilon, x + \varepsilon)) 
            &= lim_{\varepsilon \to 0} \dfrac{P[X \in (x - \varepsilon, x + \varepsilon) | Y = y]P(Y = y)}{P[X \in (x - \varepsilon, x + \varepsilon)]} \\
            &= lim_{\varepsilon \to 0} \dfrac{\dfrac{P[X \in (x - \varepsilon, x + \varepsilon) | Y = y]}{2\varepsilon}P(Y = y)}{\dfrac{P[X \in (x - \varepsilon, x + \varepsilon)]}{2\varepsilon}} \\
            &= \dfrac{f_X(x | Y = y)P(Y = y)}{f_X(x)}
        \end{aligned}$$
        So we get that $P(Y = y | X = x) = \dfrac{f_X(x | Y = y)P(Y = y)}{f_X(x)}$
\end{enumerate}

\end{homeworkProblem}

\newpage

\begin{homeworkProblem}[2]

\begin{enumerate}[(a)]
    \item As for the joint PMF of X, Y, N, we have that the PMF is $P(X = x, Y = y, N = n)$, then as we have that $N = X + Y$, then only when $x + y = n$, will the PMF be non-zero. So we have that
        $$P(X = x, Y = y, N = n) = P(X = x, Y = y) = (1 - p)^x * p * (1 - p)^y * p = (1 - p)^{x + y}p^2$$, as we have that $x + y = n$, so we get that $$P(X = x, Y = y, N = n) = (1 - p)^np^2$$
    \item As for the joint PMF os X, N, we have that the PMF is $P(X = x, N = n)$, as only when $n = x + y$ will the PMF be non-zero, so we have that
        $$P(X = x, N = n) = P(X = x, Y = n - x) = (1 - p)^xp(1 - p)^{n - x}p = (1 - p)^np^2$$
    \item As for the conditional PMF of $X$ given $N = n$, we have that the PMF is $$P(X = x | N = n) = \dfrac{P(X = x, N = n)}{P(N = n)}. $$
        The numerator is the joint PMF of $X$ and $N$, which is $P(X = x, N = n) = (1 - p)^np^2$, and the denominator is PMF of $N$, which is $P(N = n) = \sum_{x = 0}^{n}(1 - p)^np^2 = (n + 1)(1 - p)^np^2$, so we have that $$P(X = x | N = n) = \dfrac{P(X = x, N = n)}{P(N = n)} = \dfrac{(1 - p)^np^2}{(n + 1)(1 - p)^np^2} = \dfrac{1}{n + 1}. $$
        where x = 0, 1, 2, ..., n. \\
        Description: The conditional PMF of $X$ given $N = n$ is a uniform distribution, which is $P(X = x | N = n) = \dfrac{1}{n + 1}$. The event $P(X = x)$ is a Geom distribution, while the event $N = n$ is actuallu a negative binomial distribution, which denote the fail times before the second success. So the conditional PMF of $X$ given $N = n$ is $\dfrac{1}{n+1}$, which denote that the first success between the first and the second success is uniformly distributed.
\end{enumerate}

\end{homeworkProblem}

\newpage

\begin{homeworkProblem}[3]
    
\begin{enumerate}[(a)]
    \item To verify that the conditional distribution of X given X > c is the same as the distribution of c + X, firstly we can find the corresponding CDF of X given X > c, which is $P(X \leq x | X > c) = \dfrac{P(c < X \leq x)}{P(X > c)} = \dfrac{F(x) - F(c)}{1 - F(c)}$. As $X \sim Expo(\lambda)$, so we have $F(x) = 1 - e^{-\lambda x}$. So, the $P(X \leq x | X > c) = \dfrac{e^{-\lambda c} - e^{-\lambda x}}{e^{-\lambda c}} = 1 - e^{-\lambda(x - c)}. $\\
        As for the CDF of c + X, we have that $P(c + X \leq x) = P(X \leq x - c) = 1 - e^{-\lambda(x - c)}$. So we have that $P(X \leq x | X > c) = P(c + X \leq x)$. So that the conditional CDF of X given X > c is the same as the c + X. \\
    \item As for the CDF of X given X < c, we have that for x < c, $P(X \leq x | X < c) = \dfrac{P(X \leq x, X < c)}{P(X < c)} = \dfrac{P(X \leq x)}{P(X < c)} = \dfrac{1 - e^{-\lambda x}}{1 - e^{-\lambda c}}$
        As for the PDF, we have that $f(x | X < c) = (P(X \leq x | X < c))' = (\dfrac{P(X \leq x)}{P(X < c)})' = (\dfrac{1 - e^{-\lambda x}}{1 - e^{-\lambda c}})' = \dfrac{\lambda e^{-\lambda x}}{1 - e^{-\lambda c}}$ for x < c, as for $x \geq c$, PDF is zero. \\
\end{enumerate}

\end{homeworkProblem}

\newpage

\begin{homeworkProblem}[4]
    
As we have that $U_1, U_2, U_3$ be i.i.d. Unif(0, 1), and let $L = min(U_1, U_2, U_3)$, $M = max(U_1, U_2, U_3)$
\begin{enumerate}[(a)]
    \item 
        \begin{enumerate}[1.]
            \item As for the marginal CDF of M is $F_M(m) = P(M \leq m) = P(U_1 \leq m, U_2 \leq m, U_3 \leq m) = P(U_1 \leq m)P(U_2 \leq m)P(U_3 \leq m) = m^3$. For $m \in [0, 1]$\\
            \item As for the marginal PDF of M, we have that $f_M(m) = (F_M(m))' = (m^3)' = 3m^2$. For $m \in [0, 1]$\\
            \item As for the joint CDF of M and L, firstly we consider the event $L > l, M \leq m$, which is easy to calculate, that is $P(L > l, M \leq m) = (m-l)^3$, we have that $P(L \leq l, M \leq m) = P(M \leq m) - P(L > l, M \leq m) = m^3 - (m - l)^3 $ for $m \geq l$ and that $m, l \in [0, 1]$. \\
            \item As for the joint PDF of M and L, we have that $f(l, m) = \dfrac{\partial^2 P(L \leq l, M \leq m)}{\partial l\partial m} = 6(m - l)$ for $m, l \in [0, 1]$ and that $m \geq l$. \\
        \end{enumerate}
    \item As for the conditional PDF of M given L, firstly we have that $P(L > l) = P(U_1 \geq l, U_2 \geq l, U_3 \geq l) = (1 - l)^3$. Then we have that $P(L \leq l) = 1 - P(L > l) = 1 - (1 - l)^3$, then we get that $f_L(l) = 3(1 - l)^2$, where $l \in [0, 1]$. So we get $f_{M | L}(m | l) = \dfrac{f(l, m)}{f_L(l)} = \dfrac{6(m - l)}{3(1 - l)^2} = \dfrac{2(m - l)}{(1 - l)^2}$, where $m, l \in [0, 1]$ and that $m \geq l$. \\
\end{enumerate}

\end{homeworkProblem}

\newpage

\begin{homeworkProblem}[5]

Firstly, we denote that $q = 1 - p$
\begin{enumerate}[(a)]
    \item As we have that $L = min(X, Y), M = max(X, Y)$, then we have that \\
        for $l < m$, we have $P(L = l, M = m) = P(X = l, Y = m) + P(X = m, Y = l) = q^lpq^mp + q^mpq^lp = 2p^2q^{l + m}$\\
        for $l = m$, we have $P(L = l, M = m) = P(X = l, Y = l) = q^lpq^lp = p^2q^{2l}$\\
        for $l > m$, we have $P(L = l, M = m) = P(X = l, Y = m) + P(X = m, Y = l) = 0$\\
        So we get that the joint PMF of L and M is \[P(L = l, M = m) = \left\{
            \begin{array}{lr} 2p^2q^{l + m} & l < m \\ p^2q^{2l} & l = m \\ 0 & l > m 
        \end{array}\right.\]
        However, as we have that the joint PMF of X and Y depend on l and m's size, it is clearly that the joint can't be get by multiplying the function of l and m, so L and M are no independent. \\
    \item As for the marginal distribution of L
        \begin{enumerate}[(1)]
            \item By using the joint PMF, we have that 
                $$\begin{aligned}P(L = l) &= \sum_{m = l}^{\infty} P(L = l, M = m) \\ &= p^2q^{2l} + \sum_{m = l+1}^{\infty} 2p^2q^{l + m}\\
                                        &= p^2q^{2l} + 2p^2q^l\sum_{m = l+1}^{\infty}q^m \\
                                        &= p^2q^{2l} + 2p^2q^lq^{l + 1}\left(\dfrac{1 - q^{\infty}}{1 - q}\right) \\
                                        &= p^2q^{2l} + 2p^2q^lq^{l + 1}\dfrac{1}{p} \\
                                        &= p^2q^{2l} + 2pq^{2l + 1} \\
                \end{aligned}$$
            \item Also we we can use story, that is, consider 2 independent Bernuoilli independent sequence, at the time n, at least one of the 2 sequences has a success, 
                then, we have that the probability of the event is $1 - (1 - p)^2$, so we have that $P(L = l) = (1 - (1 - q^2))^l(1 - q^2) = q^{2l}(1 - q^2)$, as we have that $(p + q)^2 = 1$, so $1 - q^2 = p^2 + 2pq$, so the $P(L = l) = q^{2l}(p^2 + 2pq) = p^2q^{2l} + 2pq^{2l + 1}. $
        \end{enumerate}
        So, both using joint PMF and using story, can we find the marginal distribution of L. 
    \item As we have that $L = min(X, Y), M = max(X, Y)$, then we have that $L + M = X + Y$, then $EL + EM = E(L + M) = E(X + Y) = E(X) + E(Y)$, as $X$ and $Y$ i.i.d. $Geom(p)$, we have that $E(X) = E(Y) = 2\dfrac{q}{p} = \dfrac{2q}{p}$
        Then, we find the $E(L)$, as $P(L = l) = p^2q^{2l} + 2pq^{2l + 1}$, so 
        $$\begin{aligned}E(L) &= \sum_{l = 0}^{\infty}lP(L = l) \\
            &= (p^2 + 2pq)\sum_{l = 1}^{\infty} lq^{2l}\\
            q^2 E(L) &= (p^2 + 2pq)\sum_{l = 1}^{\infty} lq^{2l + 2} \\
            E(L) - q^2E(l) &= (p^2 + 2pq)\sum_{l = 1}^{\infty} lq^{2l + 2} \\
                &= (1 - q^2) \dfrac{q^2}{1 - q^2} \\
            E(L) &= \dfrac{q^2}{1 - q^2} \\
        \end{aligned}$$
        So we get that $EM = \dfrac{2q}{p} - EL = \dfrac{2q}{p} - \dfrac{q^2}{1 - q^2} = \dfrac{(1 - p)(3 - p)}{p(2 - p)}$
    \item As for the joint PMF of $L$ and $M - L$, we have that for $k \geq 0$
        $$P(L = l, M - L = k) = P(L = l, M = k + l), $$ when $k > 0$, we have 
        $$\begin{aligned} P(L = l, M - L = k) 
            &= P(X = l, Y = k + l) + P(X = k + l, Y = l) \\ 
            &= q^lpq^{k + l}p + q^{k + l}pq^lp \\
            &= 2p^2q^{2l + k}. 
        \end{aligned}$$
        When $k = 0$, we have that $P(L = l, M - L = k) = p^2q^{2l + k}$. \\
        As for the $P(L = l)$, we have $P(L = l) = (1 - q^2)q^{2l}$ \\
        As for $P(M - L = k)$ For $k > 0$, we have that $$P(M - L = k) = \sum_{l = 0}^{\infty} P(L = l, M = l + k)P(L = l) = \dfrac{2p^2q^k}{1 - q^2}, $$ where $P(M - L = k)P(L = l) = P(L = l, M - L = k)$, $L$ and $M - L$ are independent \\
        and for $k = 0$, we have that $P(M - L = k) = P(M = L) = \sum_{l = 0}^{\infty}P(M = L = l) = \dfrac{p^2q^k}{1 - q^2}$. \\
        So we have for k = 0, $P(L = l, M - l = k) = P(M - L = k)P(L = l)$. \\
        In conclusion, L and M - L are independent. 
\end{enumerate}

\end{homeworkProblem}

\end{document}
