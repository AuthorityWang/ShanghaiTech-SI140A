\documentclass{article}

\usepackage{fancyhdr}
\usepackage{extramarks}
\usepackage{amsmath}
\usepackage{amsthm}
\usepackage{amsfonts}
\usepackage{tikz}
\usepackage[plain]{algorithm}
\usepackage{algpseudocode}
\usepackage{enumerate}
\usepackage{tikz}

\usetikzlibrary{automata,positioning}

%
% Basic Document Settings
%  

\topmargin=-0.45in
\evensidemargin=0in
\oddsidemargin=0in
\textwidth=6.5in
\textheight=9.0in
\headsep=0.25in

\linespread{1.1}

\pagestyle{fancy}
\lhead{\hmwkAuthorName}
\chead{\hmwkClass : \hmwkTitle}
\rhead{\firstxmark}
\lfoot{\lastxmark}
\cfoot{\thepage}

\renewcommand\headrulewidth{0.4pt}
\renewcommand\footrulewidth{0.4pt}

\setlength\parindent{0pt}

%
% Create Problem Sections
%

\newcommand{\enterProblemHeader}[1]{
    \nobreak\extramarks{}{Problem \arabic{#1} continued on next page\ldots}\nobreak{}
    \nobreak\extramarks{Problem \arabic{#1} (continued)}{Problem \arabic{#1} continued on next page\ldots}\nobreak{}
}

\newcommand{\exitProblemHeader}[1]{
    \nobreak\extramarks{Problem \arabic{#1} (continued)}{Problem \arabic{#1} continued on next page\ldots}\nobreak{}
    \stepcounter{#1}
    \nobreak\extramarks{Problem \arabic{#1}}{}\nobreak{}
}

\newcommand*\circled[1]{\tikz[baseline=(char.base)]{
		\node[shape=circle,draw,inner sep=2pt] (char) {#1};}}


\setcounter{secnumdepth}{0}
\newcounter{partCounter}
\newcounter{homeworkProblemCounter}
\setcounter{homeworkProblemCounter}{1}
\nobreak\extramarks{Problem \arabic{homeworkProblemCounter}}{}\nobreak{}

%
% Homework Problem Environment
%
% This environment takes an optional argument. When given, it will adjust the
% problem counter. This is useful for when the problems given for your
% assignment aren't sequential. See the last 3 problems of this template for an
% example.
%

\newenvironment{homeworkProblem}[1][-1]{
    \ifnum#1>0
        \setcounter{homeworkProblemCounter}{#1}
    \fi
    \section{Problem \arabic{homeworkProblemCounter}}
    \setcounter{partCounter}{1}
    \enterProblemHeader{homeworkProblemCounter}
}{
    \exitProblemHeader{homeworkProblemCounter}
}

%
% Homework Details
%   - Title
%   - Class
%   - Due date
%   - Name
%   - Student ID

\newcommand{\hmwkTitle}{Homework\ \#09}
\newcommand{\hmwkClass}{Probability \& Statistics for EECS}
\newcommand{\hmwkDueDate}{Apr 16, 2023}
\newcommand{\hmwkAuthorName}{Wang Penghao}
\newcommand{\hmwkAuthorID}{2021533138}


%
% Title Page
%

\title{
    \vspace{2in}
    \textmd{\textbf{\hmwkClass:\\  \hmwkTitle}}\\
    \normalsize\vspace{0.1in}\small{Due\ on\ \hmwkDueDate\ at 23:59}\\
	\vspace{4in}
}

\author{
	Name: \textbf{\hmwkAuthorName} \\
	Student ID: \hmwkAuthorID}
\date{}

\renewcommand{\part}[1]{\textbf{\large Part \Alph{partCounter}}\stepcounter{partCounter}\\}

%
% Various Helper Commands
%

% Useful for algorithms
\newcommand{\alg}[1]{\textsc{\bfseries \footnotesize #1}}
% For derivatives
\newcommand{\deriv}[1]{\frac{\mathrm{d}}{\mathrm{d}x} (#1)}
% For partial derivatives
\newcommand{\pderiv}[2]{\frac{\partial}{\partial #1} (#2)}
% Integral dx
\newcommand{\dx}{\mathrm{d}x}
% Alias for the Solution section header
\newcommand{\solution}{\textbf{\large Solution}}
% Probability commands: Expectation, Variance, Covariance, Bias
\newcommand{\E}{\mathrm{E}}
\newcommand{\Var}{\mathrm{Var}}
\newcommand{\Cov}{\mathrm{Cov}}
\newcommand{\Bias}{\mathrm{Bias}}

\begin{document}

\maketitle

\pagebreak

\begin{homeworkProblem}[1]

\end{homeworkProblem}

\newpage

\begin{homeworkProblem}[2]

\begin{enumerate}
    \item As for the joint PMF of X, Y, N, we have that the PMF is $P(X = x, Y = y, N = n)$, then as we have that $N = X + Y$, then only when $x + y = n$, will the PMF be non-zero. So we have that
        $$P(X = x, Y = y, N = n) = P(X = x, Y = y) = (1 - p)^x * p * (1 - p)^y * p = (1 - p)^{x + y}p^2$$, as we have that $x + y = n$, so we get that $$P(X = x, Y = y, N = n) = (1 - p)^np^2$$
    \item As for the joint PMF os X, N, we have that the PMF is $P(X = x, N = n)$, as only when $n = x + y$ will the PMF be non-zero, so we have that
        $$P(X = x, N = n) = P(X = x, Y = n - x) = (1 - p)^xp(1 - p)^{n - x}p = (1 - p)^np^2$$
    \item As for the conditional PMF of $X$ given $N = n$, we have that the PMF is $$P(X = x | N = n) = \dfrac{P(X = x, N = n)}{P(N = n)}. $$
        The numerator is the joint PMF of $X$ and $N$, which is $P(X = x, N = n) = (1 - p)^np^2$, and the denominator is PMF of $N$, which is $P(N = n) = \sum_{x = 0}^{n}(1 - p)^np^2 = (n + 1)(1 - p)^np^2$, so we have that $$P(X = x | N = n) = \dfrac{P(X = x, N = n)}{P(N = n)} = \dfrac{(1 - p)^np^2}{(n + 1)(1 - p)^np^2} = \dfrac{1}{n + 1}. $$
        where x = 0, 1, 2, ..., n. \\
        Description: The conditional PMF of $X$ given $N = n$ is a uniform distribution, which is $P(X = x | N = n) = \dfrac{1}{n + 1}$. The event $P(X = x)$ is a Geom distribution, while the event $N = n$ is actuallu a negative binomial distribution, which denote the fail times before the second success. So the conditional PMF of $X$ given $N = n$ is $\dfrac{1}{n+1}$, which denote that the first success between the first and the second success is uniformly distributed.
\end{enumerate}

\end{homeworkProblem}

\newpage

\begin{homeworkProblem}[3]
    
\end{homeworkProblem}

\newpage

\begin{homeworkProblem}[4]
    
\end{homeworkProblem}

\newpage

\begin{homeworkProblem}[5]
    
\end{homeworkProblem}

\end{document}
