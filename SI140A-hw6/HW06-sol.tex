\documentclass{article}

\usepackage{fancyhdr}
\usepackage{extramarks}
\usepackage{amsmath}
\usepackage{amsthm}
\usepackage{amsfonts}
\usepackage{tikz}
\usepackage[plain]{algorithm}
\usepackage{algpseudocode}
\usepackage{enumerate}
\usepackage{tikz}

\usetikzlibrary{automata,positioning}

%
% Basic Document Settings
%  

\topmargin=-0.45in
\evensidemargin=0in
\oddsidemargin=0in
\textwidth=6.5in
\textheight=9.0in
\headsep=0.25in

\linespread{1.1}

\pagestyle{fancy}
\lhead{\hmwkAuthorName}
\chead{\hmwkClass : \hmwkTitle}
\rhead{\firstxmark}
\lfoot{\lastxmark}
\cfoot{\thepage}

\renewcommand\headrulewidth{0.4pt}
\renewcommand\footrulewidth{0.4pt}

\setlength\parindent{0pt}

%
% Create Problem Sections
%

\newcommand{\enterProblemHeader}[1]{
    \nobreak\extramarks{}{Problem \arabic{#1} continued on next page\ldots}\nobreak{}
    \nobreak\extramarks{Problem \arabic{#1} (continued)}{Problem \arabic{#1} continued on next page\ldots}\nobreak{}
}

\newcommand{\exitProblemHeader}[1]{
    \nobreak\extramarks{Problem \arabic{#1} (continued)}{Problem \arabic{#1} continued on next page\ldots}\nobreak{}
    \stepcounter{#1}
    \nobreak\extramarks{Problem \arabic{#1}}{}\nobreak{}
}

\newcommand*\circled[1]{\tikz[baseline=(char.base)]{
		\node[shape=circle,draw,inner sep=2pt] (char) {#1};}}


\setcounter{secnumdepth}{0}
\newcounter{partCounter}
\newcounter{homeworkProblemCounter}
\setcounter{homeworkProblemCounter}{1}
\nobreak\extramarks{Problem \arabic{homeworkProblemCounter}}{}\nobreak{}

%
% Homework Problem Environment
%
% This environment takes an optional argument. When given, it will adjust the
% problem counter. This is useful for when the problems given for your
% assignment aren't sequential. See the last 3 problems of this template for an
% example.
%

\newenvironment{homeworkProblem}[1][-1]{
    \ifnum#1>0
        \setcounter{homeworkProblemCounter}{#1}
    \fi
    \section{Problem \arabic{homeworkProblemCounter}}
    \setcounter{partCounter}{1}
    \enterProblemHeader{homeworkProblemCounter}
}{
    \exitProblemHeader{homeworkProblemCounter}
}

%
% Homework Details
%   - Title
%   - Class
%   - Due date
%   - Name
%   - Student ID

\newcommand{\hmwkTitle}{Homework\ \#06}
\newcommand{\hmwkClass}{Probability \& Statistics for EECS}
\newcommand{\hmwkDueDate}{Mar 26, 2023}
\newcommand{\hmwkAuthorName}{Penghao Wang}
\newcommand{\hmwkAuthorID}{2021533138}


%
% Title Page
%

\title{
    \vspace{2in}
    \textmd{\textbf{\hmwkClass:\\  \hmwkTitle}}\\
    \normalsize\vspace{0.1in}\small{Due\ on\ \hmwkDueDate\ at 23:59}\\
	\vspace{4in}
}

\author{
	Name: \textbf{\hmwkAuthorName} \\
	Student ID: \hmwkAuthorID}
\date{}

\renewcommand{\part}[1]{\textbf{\large Part \Alph{partCounter}}\stepcounter{partCounter}\\}

%
% Various Helper Commands
%

% Useful for algorithms
\newcommand{\alg}[1]{\textsc{\bfseries \footnotesize #1}}
% For derivatives
\newcommand{\deriv}[1]{\frac{\mathrm{d}}{\mathrm{d}x} (#1)}
% For partial derivatives
\newcommand{\pderiv}[2]{\frac{\partial}{\partial #1} (#2)}
% Integral dx
\newcommand{\dx}{\mathrm{d}x}
% Alias for the Solution section header
\newcommand{\solution}{\textbf{\large Solution}}
% Probability commands: Expectation, Variance, Covariance, Bias
\newcommand{\E}{\mathrm{E}}
\newcommand{\Var}{\mathrm{Var}}
\newcommand{\Cov}{\mathrm{Cov}}
\newcommand{\Bias}{\mathrm{Bias}}

\begin{document}

\maketitle

\pagebreak

\begin{homeworkProblem}[1]

Firstly we denote an indicator $I_j$, which is 1 if the $j$-th type toy is selected, and 0 otherwise. Then we denote that the total number of distinct toy types is X, 
we then have $$X = \sum_{j = 1}^{n} I_j.$$
We then find the $E(X)$, which is $$E(X) = E(\sum_{j = 1}^{n} I_j) = \sum_{j = 1}^{n}E(I_j).$$
We can denote that the probability of the $j$-th type toy is head is $p_j$, then we have $$E(I_j) = p_j.$$
So we have $$E(X) = \sum_{j = 1}^{n}p_j.$$
As $p_j = (1 - (1 - \dfrac{1}{n})^t)$, where t is the number that we totally collected toys. So we have
$$E(X) = \sum_{j = 1}^{n} (1 - (1 - \dfrac{1}{n})^t) = n(1 - (1 - \dfrac{1}{n})^t)$$

\end{homeworkProblem}

\newpage

\begin{homeworkProblem}[2]

We denote an indicator $A_i$, which is 1 if the $i$-th block not equal the $(i+1)$-th block, and 0 otherwise, and $A$ is the total number of runs, then we have that $$A = \sum_{i = 1}^{n - 1}A_i + 1.$$
So we have that $$E(A) = E(\sum_{i = 1}^{n - 1}A_i + 1) = 1 + \sum_{i = 1}^{n - 1}E(A_i).$$
Then we denote event $B_j$ is the $j th$ block is different from $j+1th$, then $P(B_j) = p(1 - p) + (1 - p)p = 2p(1 - p). $
So we have that $E(A) = 1 + \sum_{i = 1}^{n - 1} P(B_i) = 1 + 2(n-1)p(1 - p). $

\end{homeworkProblem}

\newpage

\begin{homeworkProblem}[3]
    
\end{homeworkProblem}

\newpage

\begin{homeworkProblem}[4]

\end{homeworkProblem}

\newpage

\begin{homeworkProblem}[5]

\end{homeworkProblem}

\newpage

\end{document}
