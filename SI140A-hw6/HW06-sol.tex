\documentclass{article}

\usepackage{fancyhdr}
\usepackage{extramarks}
\usepackage{amsmath}
\usepackage{amsthm}
\usepackage{amsfonts}
\usepackage{tikz}
\usepackage[plain]{algorithm}
\usepackage{algpseudocode}
\usepackage{enumerate}
\usepackage{tikz}

\usetikzlibrary{automata,positioning}

%
% Basic Document Settings
%  

\topmargin=-0.45in
\evensidemargin=0in
\oddsidemargin=0in
\textwidth=6.5in
\textheight=9.0in
\headsep=0.25in

\linespread{1.1}

\pagestyle{fancy}
\lhead{\hmwkAuthorName}
\chead{\hmwkClass : \hmwkTitle}
\rhead{\firstxmark}
\lfoot{\lastxmark}
\cfoot{\thepage}

\renewcommand\headrulewidth{0.4pt}
\renewcommand\footrulewidth{0.4pt}

\setlength\parindent{0pt}

%
% Create Problem Sections
%

\newcommand{\enterProblemHeader}[1]{
    \nobreak\extramarks{}{Problem \arabic{#1} continued on next page\ldots}\nobreak{}
    \nobreak\extramarks{Problem \arabic{#1} (continued)}{Problem \arabic{#1} continued on next page\ldots}\nobreak{}
}

\newcommand{\exitProblemHeader}[1]{
    \nobreak\extramarks{Problem \arabic{#1} (continued)}{Problem \arabic{#1} continued on next page\ldots}\nobreak{}
    \stepcounter{#1}
    \nobreak\extramarks{Problem \arabic{#1}}{}\nobreak{}
}

\newcommand*\circled[1]{\tikz[baseline=(char.base)]{
		\node[shape=circle,draw,inner sep=2pt] (char) {#1};}}


\setcounter{secnumdepth}{0}
\newcounter{partCounter}
\newcounter{homeworkProblemCounter}
\setcounter{homeworkProblemCounter}{1}
\nobreak\extramarks{Problem \arabic{homeworkProblemCounter}}{}\nobreak{}

%
% Homework Problem Environment
%
% This environment takes an optional argument. When given, it will adjust the
% problem counter. This is useful for when the problems given for your
% assignment aren't sequential. See the last 3 problems of this template for an
% example.
%

\newenvironment{homeworkProblem}[1][-1]{
    \ifnum#1>0
        \setcounter{homeworkProblemCounter}{#1}
    \fi
    \section{Problem \arabic{homeworkProblemCounter}}
    \setcounter{partCounter}{1}
    \enterProblemHeader{homeworkProblemCounter}
}{
    \exitProblemHeader{homeworkProblemCounter}
}

%
% Homework Details
%   - Title
%   - Class
%   - Due date
%   - Name
%   - Student ID

\newcommand{\hmwkTitle}{Homework\ \#06}
\newcommand{\hmwkClass}{Probability \& Statistics for EECS}
\newcommand{\hmwkDueDate}{Mar 26, 2023}
\newcommand{\hmwkAuthorName}{Penghao Wang}
\newcommand{\hmwkAuthorID}{2021533138}


%
% Title Page
%

\title{
    \vspace{2in}
    \textmd{\textbf{\hmwkClass:\\  \hmwkTitle}}\\
    \normalsize\vspace{0.1in}\small{Due\ on\ \hmwkDueDate\ at 23:59}\\
	\vspace{4in}
}

\author{
	Name: \textbf{\hmwkAuthorName} \\
	Student ID: \hmwkAuthorID}
\date{}

\renewcommand{\part}[1]{\textbf{\large Part \Alph{partCounter}}\stepcounter{partCounter}\\}

%
% Various Helper Commands
%

% Useful for algorithms
\newcommand{\alg}[1]{\textsc{\bfseries \footnotesize #1}}
% For derivatives
\newcommand{\deriv}[1]{\frac{\mathrm{d}}{\mathrm{d}x} (#1)}
% For partial derivatives
\newcommand{\pderiv}[2]{\frac{\partial}{\partial #1} (#2)}
% Integral dx
\newcommand{\dx}{\mathrm{d}x}
% Alias for the Solution section header
\newcommand{\solution}{\textbf{\large Solution}}
% Probability commands: Expectation, Variance, Covariance, Bias
\newcommand{\E}{\mathrm{E}}
\newcommand{\Var}{\mathrm{Var}}
\newcommand{\Cov}{\mathrm{Cov}}
\newcommand{\Bias}{\mathrm{Bias}}

\begin{document}

\maketitle

\pagebreak

\begin{homeworkProblem}[1]

Firstly we denote an indicator $I_j$, which is 1 if the $j$-th type toy is selected, and 0 otherwise. Then we denote that the total number of distinct toy types is X, 
we then have $$X = \sum_{j = 1}^{n} I_j.$$
We then find the $E(X)$, which is $$E(X) = E(\sum_{j = 1}^{n} I_j) = \sum_{j = 1}^{n}E(I_j).$$
We can denote that the probability of the $j$-th type toy is head is $p_j$, then we have $$E(I_j) = p_j.$$
So we have $$E(X) = \sum_{j = 1}^{n}p_j.$$
As $p_j = (1 - (1 - \dfrac{1}{n})^t)$, where t is the number that we totally collected toys. So we have
$$E(X) = \sum_{j = 1}^{n} (1 - (1 - \dfrac{1}{n})^t) = n(1 - (1 - \dfrac{1}{n})^t)$$

\end{homeworkProblem}

\newpage

\begin{homeworkProblem}[2]

We denote an indicator $A_i$, which is 1 if the $i$-th block not equal the $(i+1)$-th block, and 0 otherwise, and $A$ is the total number of runs, then we have that $$A = \sum_{i = 1}^{n - 1}A_i + 1.$$
So we have that $$E(A) = E(\sum_{i = 1}^{n - 1}A_i + 1) = 1 + \sum_{i = 1}^{n - 1}E(A_i).$$
Then we denote event $B_j$ is the $j th$ block is different from $j+1th$, then $P(B_j) = p(1 - p) + (1 - p)p = 2p(1 - p). $
So we have that $E(A) = 1 + \sum_{i = 1}^{n - 1} P(B_i) = 1 + 2(n-1)p(1 - p). $

\end{homeworkProblem}

\newpage

\begin{homeworkProblem}[3]
    
\begin{enumerate}
    \item Fristly we need to consider let the last one be the tagged elk, then we need to make k-1 captures. We have
        $$\begin{aligned} P(X = k) &= \dfrac{\begin{pmatrix} n \\ m-1 \end{pmatrix}\begin{pmatrix} N-n \\ k \end{pmatrix}}{\begin{pmatrix} N \\ m+k-1 \end{pmatrix}} * \dfrac{n - (m - 1)}{N - (m - 1) - k} \\
                                   &= \dfrac{\begin{pmatrix} m + k - 1 \\ m - 1 \end{pmatrix}\begin{pmatrix}N - m -k \\ n - m\end{pmatrix}}{\begin{pmatrix}N \\ n\end{pmatrix}} \end{aligned}.$$
        Then the total number of elk in the new sample, as $Y = X + m$, then we have that $$P(Y = y) = P(X = y - m) = \dfrac{\begin{pmatrix} n \\ m - 1 \end{pmatrix}\begin{pmatrix}N - n \\ y - m \end{pmatrix}}{\begin{pmatrix} N \\ y - 1 \end{pmatrix}} * \dfrac{n - m + 1}{N - y + 1}$$
    \item Define that untaggd elk are labeled 1, 2, ..., N - n, then we define that $X_1, X_2, ..., X_m$ are the number of untaggd elk before the first tagged elk, then number between first and second tagged elk, ..., then we have that $X_1 = I_1+ ... + I_{N - n}$, $I_j$ is the indicator of untagged elk j that is captured before tagged elk. 
        Then we have that $E(I_j) = \dfrac{1}{n+1}$, so we have $E(X_1) = \dfrac{N - n}{n + 1}$, then we have that $E(X_j) = \dfrac{N-n}{n+1}$ for all $j = 1, 2,..., m$. So we have 
        $$E(X) = \dfrac{m(N - n)}{n + 1}$$, as $Y = X+m$, we have $$E(Y) = E(X + m) = \dfrac{m(N + 1)}{n+1}. $$
    \item Suppose that $E[Y]$ is an integer, and sample with fixed size, we have that the number of tagged elk is distribute to hypergeometric distribution, and we deonte that tagged elk in the sample is $Z$, we have that $E[Z] = \dfrac{m(N + 1)}{n+1}*\dfrac{n}{N} = m * \dfrac{1 + \dfrac{1}{N}}{1 + \dfrac{1}{n}} < m$
    So is less than m. 
\end{enumerate}

\end{homeworkProblem}

\newpage

\begin{homeworkProblem}[4]

\begin{enumerate}
    \item Firstly from the question, we know that 23 is the minimum integer that makes chance of birthday match larger than 50, then we need to calculate the $P(X \leq 23) = 1 - P(X \geq 24)$. 
    As this is distribute to possion distribution, then we have that $P(X \geq k) = e^{\dfrac{-(k - 1)^2}{365 * 2}}$. So we get that $P(X \geq 24) \textless \dfrac{1}{2}. $
    So, we can get that $P(X \leq 23) \geq \dfrac{1}{2}. $ So we get that 23 is the median of X, then as for unique, we have that with k get larger, $P(X \geq k)$ becomes smaller, then $P(X \leq k) \leq P(X \leq 22) \textless \dfrac{1}{2}$, $P(X \geq k) \geq P(X \geq 24) \textgreater \dfrac{1}{2}$, so we have that 23 is the unique mean of X. 
    \item As $I_j$ is the indicator random variable for the event $X \geq j$, then suppose that $X = k$, then we have that as for $I_1 + I_2 + ... + I_{366} = I_1 + I_2 + ... + I_j + 0 = j = X$, so we have that $X = I_1 + I_2 + ... + I_{366}$, then we have that as for $P(X \geq j) = \dfrac{365 * ... * (365 + 2 - j)}{365^{j - 1}} = (1 - \dfrac{1}{365})(1 - \dfrac{2}{365})...(1 - \dfrac{j - 2}{365}) = p_j$. 
    Then as we have that $E(X) = E(I_1 + ... + I_366) = E(I_1) + E(I_2) + ... + E(I_{366}) = P(X \geq 1) + ... + P(X \geq 366) = \sum_{j = 1}^{366}p_j$. So we get that $E(X) = \sum_{j = 1}^{366} p_j$
    \item As $E(X) = \sum_{j = 1}^{366}p_j$, then as $p_j = (1 - \dfrac{1}{365})(1 - \dfrac{2}{365})...(1 - \dfrac{j-2}{365})$, then with numerically calculation, we get that $E(X) = 24.6165859$.
    \item Firstly we use $p_j$ to implement this, we have that $Var(X) = E(X^2) - [E(X)]^2$, then firstly, $X^2 = \sum_{j = 1}^{366}I_j^2 + 2\sum_{i = 1}^{365}\sum_{j = i + 1}^{366}I_iI_j$. Then as $I_j^2 = I_j$ and that $I_iI_j = I_j$ with that $i \textless j$, then we have that $E(X^2) = \sum_{j = 1}^{366}I_j + 2\sum_{j = 2}^{366}(j - 1)I_j$. 
    So we have that $Var(X) = E(X^2) - [E(x)]^2 = \sum_{j = 1}^{366}p_j + 2\sum_{j = 2}^{366}(j - 1)p_j - (\sum_{j = 1}^{366}p_j)^2$. Then we use numerically calculation, we get that $Var(X) = 148.6402848. $
    So, we get that $Var(X) = \sum_{j = 1}^{366}p_j + 2\sum_{j = 2}^{366}(j - 1)p_j - (\sum_{j = 1}^{366}p_j)^2$ and that $Var(X) = 148.6402848.$
\end{enumerate}

\end{homeworkProblem}

\newpage

\begin{homeworkProblem}[5]

To distribute the 14 balls into 5 boxes, we firstly denote event A be all the cases to put 14 balls into 5 boxes, B be the event all the cases to put 14 balls into boxes, with striction that one can at most put balls. Event $C_i$ be that the $i-th$ box has balls more than 6. Then we have
$$B^c = \cup_{i = 1}^5 C_i.$$
Then we have with inclusion-exclusion, 
$$\begin{aligned}
    P(B^c) &= \sum_{i}P(C_i) - \sum_{i < j} P(C_i \cap C_j) + ... + (-1)^{5+1}P(C_1 \cap ... \cap C_5)
\end{aligned}$$
As for $P(C_i)$, we have that firstly take 7 balls into the i th box, then distribute the left 7 balls into 5 boxes, that is $P(C_i)$
= $\begin{pmatrix} 11 \\ 4 \end{pmatrix}$. Then as for $P(C_i \cap C_j)$, we have that distribute 7 balls into a box and the left 7 balls into another box, that is only 1 cases. \\
So we have that $$\begin{aligned}P(B^c) &= \sum_{i}P(C_i) - \sum_{i < j} P(C_i \cap C_j) + ... + (-1)^{5+1}P(C_1 \cap ... \cap C_5)\\ 
    &= \dfrac{5 * \begin{pmatrix}11 \\ 4\end{pmatrix} - 10 * 1 + 0 - 0 + 0}{\begin{pmatrix} 18 \\ 4 \end{pmatrix}} \\
    &= \dfrac{1640}{3060}. \end{aligned}$$
So we have that ways of B is $A - B^c = 3060 - 1640 = 1420. $\\
So there are totally 1420 ways to distribute 14 balls into 5 boxes, with the restriction that one box can at most put 6 balls.
\end{homeworkProblem}

\newpage

\end{document}
