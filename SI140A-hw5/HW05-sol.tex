\documentclass{article}

\usepackage{fancyhdr}
\usepackage{extramarks}
\usepackage{amsmath}
\usepackage{amsthm}
\usepackage{amsfonts}
\usepackage{tikz}
\usepackage[plain]{algorithm}
\usepackage{algpseudocode}
\usepackage{enumerate}
\usepackage{tikz}

\usetikzlibrary{automata,positioning}

%
% Basic Document Settings
%  

\topmargin=-0.45in
\evensidemargin=0in
\oddsidemargin=0in
\textwidth=6.5in
\textheight=9.0in
\headsep=0.25in

\linespread{1.1}

\pagestyle{fancy}
\lhead{\hmwkAuthorName}
\chead{\hmwkClass : \hmwkTitle}
\rhead{\firstxmark}
\lfoot{\lastxmark}
\cfoot{\thepage}

\renewcommand\headrulewidth{0.4pt}
\renewcommand\footrulewidth{0.4pt}

\setlength\parindent{0pt}

%
% Create Problem Sections
%

\newcommand{\enterProblemHeader}[1]{
    \nobreak\extramarks{}{Problem \arabic{#1} continued on next page\ldots}\nobreak{}
    \nobreak\extramarks{Problem \arabic{#1} (continued)}{Problem \arabic{#1} continued on next page\ldots}\nobreak{}
}

\newcommand{\exitProblemHeader}[1]{
    \nobreak\extramarks{Problem \arabic{#1} (continued)}{Problem \arabic{#1} continued on next page\ldots}\nobreak{}
    \stepcounter{#1}
    \nobreak\extramarks{Problem \arabic{#1}}{}\nobreak{}
}

\newcommand*\circled[1]{\tikz[baseline=(char.base)]{
		\node[shape=circle,draw,inner sep=2pt] (char) {#1};}}


\setcounter{secnumdepth}{0}
\newcounter{partCounter}
\newcounter{homeworkProblemCounter}
\setcounter{homeworkProblemCounter}{1}
\nobreak\extramarks{Problem \arabic{homeworkProblemCounter}}{}\nobreak{}

%
% Homework Problem Environment
%
% This environment takes an optional argument. When given, it will adjust the
% problem counter. This is useful for when the problems given for your
% assignment aren't sequential. See the last 3 problems of this template for an
% example.
%

\newenvironment{homeworkProblem}[1][-1]{
    \ifnum#1>0
        \setcounter{homeworkProblemCounter}{#1}
    \fi
    \section{Problem \arabic{homeworkProblemCounter}}
    \setcounter{partCounter}{1}
    \enterProblemHeader{homeworkProblemCounter}
}{
    \exitProblemHeader{homeworkProblemCounter}
}

%
% Homework Details
%   - Title
%   - Class
%   - Due date
%   - Name
%   - Student ID

\newcommand{\hmwkTitle}{Homework\ \#05}
\newcommand{\hmwkClass}{Probability \& Statistics for EECS}
\newcommand{\hmwkDueDate}{Mar 19, 2023}
\newcommand{\hmwkAuthorName}{Penghao Wang}
\newcommand{\hmwkAuthorID}{2021533138}


%
% Title Page
%

\title{
    \vspace{2in}
    \textmd{\textbf{\hmwkClass:\\  \hmwkTitle}}\\
    \normalsize\vspace{0.1in}\small{Due\ on\ \hmwkDueDate\ at 23:59}\\
	\vspace{4in}
}

\author{
	Name: \textbf{\hmwkAuthorName} \\
	Student ID: \hmwkAuthorID}
\date{}

\renewcommand{\part}[1]{\textbf{\large Part \Alph{partCounter}}\stepcounter{partCounter}\\}

%
% Various Helper Commands
%

% Useful for algorithms
\newcommand{\alg}[1]{\textsc{\bfseries \footnotesize #1}}
% For derivatives
\newcommand{\deriv}[1]{\frac{\mathrm{d}}{\mathrm{d}x} (#1)}
% For partial derivatives
\newcommand{\pderiv}[2]{\frac{\partial}{\partial #1} (#2)}
% Integral dx
\newcommand{\dx}{\mathrm{d}x}
% Alias for the Solution section header
\newcommand{\solution}{\textbf{\large Solution}}
% Probability commands: Expectation, Variance, Covariance, Bias
\newcommand{\E}{\mathrm{E}}
\newcommand{\Var}{\mathrm{Var}}
\newcommand{\Cov}{\mathrm{Cov}}
\newcommand{\Bias}{\mathrm{Bias}}

\begin{document}

\maketitle

\pagebreak

\begin{homeworkProblem}[1]

\begin{enumerate}
    \item As the treasure has equally probability to be in the realm from 1 to 9. Then we have that: \\
    \begin{enumerate}
        \item If treasure in realm 1, then we need to ask 1 questions. \\
        \item If treasure in realm 2, then we need to ask 2 questions. \\
        \item ...
        \item If treasure in realm 8, then we need to ask 8 questions. \\
        \item Note that if treasure in realm 9, then we need to ask 8 questions. \\
    \end{enumerate}
    Then we denote that event $X_i$ is that treasure in realm $i$. Then we have that: \\
    $$P(X_i) = \dfrac{1}{9}.$$
    Then we denote that $Y$ is the number of questions we need to ask. Then we have that: \\
    $$P(Y = k) = P(X_k) = \dfrac{1}{9}$$
    Then we have that: 
    $$E(X_i) = 1 * \dfrac{1}{9} + 2 * \dfrac{1}{9} + ... + 8 * \dfrac{2}{9} = \dfrac{44}{9}. $$
    \item 
    By using the bisection method, we could calculate the expected questions for each case that the treasure is in a specific realm. 
    \begin{enumerate}
        \item If treasure in realm 1, then we need to ask 4 questions. \\
        \item If treasure in realm 2, then we need to ask 4 questions. \\
        \item If treasure in realm 3, then we need to ask 3 questions. \\
        \item If treasure in realm 4, then we need to ask 3 questions. \\
        \item If treasure in realm 5, then we need to ask 3 questions. \\
        \item If treasure in realm 6, then we need to ask 3 questions. \\
        \item If treasure in realm 7, then we need to ask 3 questions. \\
        \item If treasure in realm 8, then we need to ask 3 questions. \\
        \item If treasure in realm 9, then we need to ask 3 questions. \\
    \end{enumerate}
    Then we have that: the expected questions is $\dfrac{1}{9} * 4 + ... + \dfrac{1}{9} * 3 = \dfrac{29}{9}$. 
\end{enumerate}

\end{homeworkProblem}

\newpage

\begin{homeworkProblem}[2]

Use the model Coupon Collector in Lecture, we firstly denote that $X_k$ is the number of days needed to see the Youthuber eating steaks with $k$ types deneness at least once. \\
Firstly we have $E[X_1] = 1$, then $X_2 - X_1 \sim Geom(\dfrac{n-1}{n})$, then we have $X_k - X_{k - 1} \sim Geom(\dfrac{n - k}{n})$, then donote that $W_k = X_k - X_{k - 1}$\\
Then we have 
$$\begin{aligned}E[X_n] &= E[X_1 + X_2 - X_1 + ... + X_n - X_{n - 1}]\\ 
                       &= E[W_1 + W_2 + ... + W_n] \\
                       &= E[W_1] + E[W_2] + E[W_n] \\
                       &= 1 + \dfrac{n}{n-1}+ ... + \dfrac{n}{1} \\
                       &= n\sum_{i = 1}^{n} \dfrac{1}{i}
\end{aligned}$$
In this problem, k = 5, then we have $E[X_5] = 5 * (\dfrac{1}{1} + \dfrac{1}{2} + ... + \dfrac{1}{5}) = \dfrac{137}{12}.$
\end{homeworkProblem}

\newpage

\begin{homeworkProblem}[3]

\begin{enumerate}
    \item Denote the times that they are simulataneously successful is $A$. As we looking for the cases that they are simultaneously successful, where the probability that simulataneously successful is $p_1 * p_2 = p_1 p_2$, then this is the geometric distribution, we have $$A-1 \sim Geom(p_1p_2)$$ and $$A \sim Fs(p_1p_2)$$ 
    Then we denote that $p = p_1p_2$ and that $q = 1-p$, we have : $$E(A) = \sum_{k = 0} ^ \infty kP(X = k) = \sum_{k = 0} ^ \infty kq^{k-1}p = \dfrac{p}{q}\sum_{k = 0} ^ \infty kq^k. $$
    Firstly we denote the $RES = \sum_{k = 0} ^ n kq^k$, as $q * RES = q\sum_{k = 0} ^ n kq^{k} = \sum_{k = 0} ^ \infty kq^{k + 1}$, then make subtraction and get $RES - q*RES = -nq^{n+1} + \sum_{i = 1} ^ n q^n $, so we get that $RES = \dfrac{q(1-q^n - nq^n + nq^{n+1})}{(1 - q) ^ 2}$. 
    Take it into $E(A)$, we have $E(A) = \dfrac{1}{p} \lim_{n \rightarrow \infty} (1-q^n - nq^n + nq^{n+1}) = \dfrac{1}{p}$. \\
    Then we get $p = p_1p_2$ and that $q = 1-p$ into $E(A)$, we have that $$E(A) = \dfrac{1}{p_1p_2}. $$
    \item Denote the times that at least one has a success is $B$, the event's probability is $1 - (1 - p_1)(1 - p_2) = p_1 + p_2 - p_1p_2$, then the same as this problem's question 1, we denote that $p = p_1 + p_2 - p_1p_2, q = 1 - p$, we have $$E(B) = \dfrac{1}{p} = \dfrac{1}{p_1 + p_2 - p_1p_2}. $$
    \item Firstly we denote that $p = p_1 = p_2$, and $q = 1 - p$, then we find the probability that they simultaneously success, denote that $C_1, C_2$ are the first time they success, then we have the probability $P(C_1 = C_2) = \sum_{n = 1}^{\infty}P(C_1 = n, C_2 = n) = \sum_{n = 1}^{\infty} p^2 q^{2(n - 1)} = \dfrac{p^2}{1 - q^2} = \dfrac{p}{2 - p}$. \\
    Then, as we have that $P(C_1 = C_2) + P(C_1 < C_2) + P(C_1 > C_2) = P(C_1 = C_2) + 2P(C_1 < C_2) = 1$, we get the $$P(C_1 < C_2) = \dfrac{1}{2}(1 - \dfrac{p}{2 - p}) = \dfrac{1 - p}{2 - p}. $$
\end{enumerate}

\end{homeworkProblem}

\newpage

\begin{homeworkProblem}[4]

Firstly we denote the $RES = \sum_{k = 0} ^ n kq^k$, as $q * RES = q\sum_{k = 0} ^ n kq^{k} = \sum_{k = 0} ^ \infty kq^{k + 1}$, then make subtraction and get $RES - q*RES = -nq^{n+1} + \sum_{i = 1} ^ n q^n $, so we get that $RES = \dfrac{q(1-q^n - nq^n + nq^{n+1})}{(1 - q) ^ 2}$. Then the $\lim_{n \rightarrow \infty}\sum_{k = 0} ^ n kq^k = \dfrac{q}{(1 - q)^2}$. So we get that: 
$$\sum_{k = 0} ^ \infty kq^k = \dfrac{q}{(1 - q)^2}$$  

\begin{enumerate}
    \item According to the problem, we have that $X \sim Geom(p), Y \sim Geom(q)$. We have that $P(X = k) = (1 - p) ^ k p, P(Y = k) = (1 - q) ^ k q$. Then $P(X = Y) = \sum_{k = 0}^{\infty}P(X = k, Y = k) = \sum_{k = 0}^{\infty}(1 - p)^kp (1 - q)^kq = pq \sum_{k = 0}^{\infty}((1 - p)(1 - q))^k = pq \dfrac{1}{1 - (1 - p)(1 - q)} = \dfrac{pq}{p+q - pq}. $So, we have 
    $$P(X = Y) = \dfrac{pq}{p + q - pq}. $$
    \item To solve $E[max(X, Y)]$, we firstly calculate the $P(max(X, Y) = k)$. \\
    $$\begin{aligned} P[max(X, Y) = k] &= P(X = k, Y \leq k) + P(Y = k, X \textless k) \\
        &= P(X = k)P(Y \leq k) + P(Y = k)P(X \textless k) \\
        &= (1 - p)^kp \sum_{n = 0}^{k}(1 - q)^nq + (1 - q)^kq\sum_{n = 0}^{k-1}(1 - p)^np \\
        &= (1 - p)^kp[1 - (1 - q)^{k+1}] + (1 - q)^kq[1 - (1 - p)^{k}]
    \end{aligned}$$
    We have that
    $$\begin{aligned}
        E[max(X, Y)] &= \sum_{k = 0}^{\infty} kP[max(X, Y) = k] \\
                     &= \sum_{k = 0}^{\infty} k\{(1 - p)^kp[1 - (1 - q)^{k + 1}] + (1 - q)^kq[1 - (1 - p)^k]\} \\
                     &= \sum_{k = 0}^{\infty} k(1 - p)^kp - k(1 - p)^kp(1 - q)^{k + 1} + k(1 - q)^kq - k(1 - q)^kq(1 - p)^k \\
                     &= \dfrac{1 - p}{p} - p(1 - q)\dfrac{(1 - p)(1 - q)}{(p + q - pq)^2} + \dfrac{1 - q}{q} - q\dfrac{(1 - p)(1 - q)}{(p + q - pq)^2} \\
                     &= \dfrac{1 - p}{p} + \dfrac{1 - q}{q} - \dfrac{(1 - p)(1 - q)}{p + q - pq} \\
                     &= -1 - \dfrac{1}{p + q - pq} + \dfrac{1}{p} + \dfrac{1}{q}. 
    \end{aligned}$$
    \item According to the question 2 of this problem, we get that 
    $$\begin{aligned}
        P[min(X, Y) = k] &= P(X = k, Y \geq k) + P(Y = k, X \textgreater k) \\
                         &= P(X = k)P(Y \geq k) + P(Y = k)P(X \textgreater k) \\
                         &= (1 - p)^kp\sum_{n = k}^{\infty}(1 - q)^nq + (1 - q)^kq\sum_{n = {k + 1}}^{\infty}(1 - p)^np \\
                         &= (1 - p)^kp (1 - q)^k + (1 - q)^kq (1 - p)^{k + 1} \\
                         &= [p + (1 - p)q](1 - p)^k(1 - q)^k\\
                         &= (p + q - pq)[(1 - p)(1 - q)] ^ k
    \end{aligned}$$
    \item Firstly we calculate the probability $P(X = i | X \leq Y)$, we have 
    $$ P(X = i | X \leq Y) = \dfrac{P(X = i, Y \geq i)}{P(X \leq Y)} = \dfrac{(1 - p)^ip(1 - q)^i}{\sum_{j = 0}^{\infty}P(X = j)P(Y \geq j)} = \dfrac{(1 - p)^ip(1 - q)^i}{\sum_{j = 0}^{\infty}(1 - q)^j(1 - p)^jp}$$
    We get that $$P(X = i | X \leq Y) = (1 - p)^n(1 - q)^n(p + q - pq)$$
    $$\begin{aligned}
        E[X | X \leq Y] &= \sum_{k = 0}^{\infty} k P(X = k | X \leq Y) \\
                        &= \sum_{k = 0}^{\infty} k (p + q - pq)(1 - p)^k(1 - q)^k \\
                        &= (p + q - pq)\dfrac{(1 - p)(1 - q)}{(p + q - pq)^2} \\
                        &= \dfrac{(1 - p)(1 - q)}{p + q - pq}
    \end{aligned}$$
    So, the $E[X | X \leq Y] = \dfrac{(1 - p)(1 - q)}{p + q - pq}$
\end{enumerate}

\end{homeworkProblem}

\newpage

\begin{homeworkProblem}[5]

\begin{enumerate}
    \item Denote that $A_j$ is the rank of the $jth$ dish you try, $A$ is the sum of the ranks, $X_i$ is the $ith$ dish's rank. We have $A = A_1 + ... + A_k + (m - k)X$, as X is the rank of the best dish that you find in the exploration phase.
    $$\begin{aligned}
        E(A_i) &= E(Y_i | X = X_1)P(X = X_1) + ... + E(Y_i | X = X_k)P(X = X_k) \\ 
               &= \dfrac{\sum_{j = 1}^{X_1 - 1} j }{k(X_1 - 1)} + ... + \dfrac{\sum_{j = 1}^{X_k - 1} j }{k(X_k - 1)} \\
               &= \dfrac{X_1}{2k} + \dfrac{X_2}{2k} + ... + \dfrac{X_k}{2k} \\
               &= \dfrac{E(X)}{2}
    \end{aligned}$$
    We then have $$E(A) = E(A_1) + E(A_2) + ... + E(A_k) + E(X) + (m - k)E(X) = \dfrac{E(X)(k + 1)}{2} + (m - k)E(X)$$
    So we get $$E(A) = (m - \dfrac{k}{2} + \dfrac{1}{2})E(X)$$
    \item To find the PMF of $X$, which is $P(X = i)$, we need to choose a dish with rank i and other $k-1$ dishes with rank $\leq i$ from 1 to $j - 1$, we have $$P(X = i) = \dfrac{\begin{pmatrix} i - 1 \\ k - 1 \end{pmatrix}}{\begin{pmatrix} n \\ k \end{pmatrix}}. $$
    \item As we have get the PMF of $X$, then we get that
    $$E(X) = \sum_{i = k}^{n} i P(X = i) = \sum_{i = k}^{n} i \dfrac{\begin{pmatrix} i - 1 \\ k - 1 \end{pmatrix}}{\begin{pmatrix} n \\ k \end{pmatrix}}
             = \dfrac{k}{\begin{pmatrix}n \\ k\end{pmatrix}} \sum_{i = k}^{n} \begin{pmatrix} i \\ k \end{pmatrix}
             = k \dfrac{\begin{pmatrix}n+1 \\ k+1 \end{pmatrix}}{\begin{pmatrix}n \\ k \end{pmatrix}}
             = \dfrac{k(n + 1)}{k + 1}$$
    \item To make the $E[X]$ maxmimze, we have 
            $$E[X] = (m - \dfrac{k}{2} + \dfrac{1}{2})\dfrac{k(n + 1)}{k + 1}$$
        To make this maxmimize, we need to get the differential to k, then we have $$E[X]' = -\dfrac{1}{2}\dfrac{k(n + 1)}{k + 1} + (m - \dfrac{k}{2} + \dfrac{1}{2})\dfrac{n+1}{(k+1)^2} = 0$$
        Then we get $$k = \dfrac{-2 \pm \sqrt{4 - 4(-2m - 1)}}{2} = \sqrt{2(m + 1)} \pm 1$$
        As $k \in [0, m]$, we get finally k is $$k = \sqrt{2(m + 1)} - 1.$$
\end{enumerate}

\end{homeworkProblem}

\end{document}