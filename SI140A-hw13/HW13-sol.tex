\documentclass{article}

\usepackage{fancyhdr}
\usepackage{extramarks}
\usepackage{amsmath}
\usepackage{amsthm}
\usepackage{amsfonts}
\usepackage{tikz}
\usepackage[plain]{algorithm}
\usepackage{algpseudocode}
\usepackage{enumerate}
\usepackage{tikz}

\usetikzlibrary{automata,positioning}

%
% Basic Document Settings
%  

\topmargin=-0.45in
\evensidemargin=0in
\oddsidemargin=0in
\textwidth=6.5in
\textheight=9.0in
\headsep=0.25in

\linespread{1.1}

\pagestyle{fancy}
\lhead{\hmwkAuthorName}
\chead{\hmwkClass : \hmwkTitle}
\rhead{\firstxmark}
\lfoot{\lastxmark}
\cfoot{\thepage}

\renewcommand\headrulewidth{0.4pt}
\renewcommand\footrulewidth{0.4pt}

\setlength\parindent{0pt}

%
% Create Problem Sections
%

\newcommand{\enterProblemHeader}[1]{
    \nobreak\extramarks{}{Problem \arabic{#1} continued on next page\ldots}\nobreak{}
    \nobreak\extramarks{Problem \arabic{#1} (continued)}{Problem \arabic{#1} continued on next page\ldots}\nobreak{}
}

\newcommand{\exitProblemHeader}[1]{
    \nobreak\extramarks{Problem \arabic{#1} (continued)}{Problem \arabic{#1} continued on next page\ldots}\nobreak{}
    \stepcounter{#1}
    \nobreak\extramarks{Problem \arabic{#1}}{}\nobreak{}
}

\newcommand*\circled[1]{\tikz[baseline=(char.base)]{
		\node[shape=circle,draw,inner sep=2pt] (char) {#1};}}


\setcounter{secnumdepth}{0}
\newcounter{partCounter}
\newcounter{homeworkProblemCounter}
\setcounter{homeworkProblemCounter}{1}
\nobreak\extramarks{Problem \arabic{homeworkProblemCounter}}{}\nobreak{}

%
% Homework Problem Environment
%
% This environment takes an optional argument. When given, it will adjust the
% problem counter. This is useful for when the problems given for your
% assignment aren't sequential. See the last 3 problems of this template for an
% example.
%

\newenvironment{homeworkProblem}[1][-1]{
    \ifnum#1>0
        \setcounter{homeworkProblemCounter}{#1}
    \fi
    \section{Problem \arabic{homeworkProblemCounter}}
    \setcounter{partCounter}{1}
    \enterProblemHeader{homeworkProblemCounter}
}{
    \exitProblemHeader{homeworkProblemCounter}
}

%
% Homework Details
%   - Title
%   - Class
%   - Due date
%   - Name
%   - Student ID

\newcommand{\hmwkTitle}{Homework\ \#13}
\newcommand{\hmwkClass}{Probability \& Statistics for EECS}
\newcommand{\hmwkDueDate}{May 14, 2023}
\newcommand{\hmwkAuthorName}{Penghao Wang}
\newcommand{\hmwkAuthorID}{2021533138}


%
% Title Page
%

\title{
    \vspace{2in}
    \textmd{\textbf{\hmwkClass:\\  \hmwkTitle}}\\
    \normalsize\vspace{0.1in}\small{Due\ on\ \hmwkDueDate\ at 23:59}\\
	\vspace{4in}
}

\author{
	Name: \textbf{\hmwkAuthorName} \\
	Student ID: \hmwkAuthorID}
\date{}

\renewcommand{\part}[1]{\textbf{\large Part \Alph{partCounter}}\stepcounter{partCounter}\\}

%
% Various Helper Commands
%

% Useful for algorithms
\newcommand{\alg}[1]{\textsc{\bfseries \footnotesize #1}}
% For derivatives
\newcommand{\deriv}[1]{\frac{\mathrm{d}}{\mathrm{d}x} (#1)}
% For partial derivatives
\newcommand{\pderiv}[2]{\frac{\partial}{\partial #1} (#2)}
% Integral dx
\newcommand{\dx}{\mathrm{d}x}
% Alias for the Solution section header
\newcommand{\solution}{\textbf{\large Solution}}
% Probability commands: Expectation, Variance, Covariance, Bias
\newcommand{\E}{\mathrm{E}}
\newcommand{\Var}{\mathrm{Var}}
\newcommand{\Cov}{\mathrm{Cov}}
\newcommand{\Bias}{\mathrm{Bias}}

\begin{document}

\maketitle

\pagebreak

\begin{homeworkProblem}[1]

\begin{enumerate}[(a)]
    \item As for the distribution of $N$ - 1, as $X_1, X_2, ... $ i.i.d. Expo(1),
        we have that $X_1, X_2, ...$ are independent, and each $X$ has probability of 
        $\dfrac{1}{e}$ to exceed 1. As to the definition of Geometric distribution,
        which is the number of trails to get the first success, we have that
        $N$ - 1 follows Geometric distribution with parameter $\dfrac{1}{e}$.
        So we get that $$N - 1 \sim Geom(\dfrac{1}{e}), $$ then with the property of Geometric distribution,
        we have that $$E(N) = E(N - 1) + 1 = \dfrac{1 - 1 / e}{1 / e} + 1 = e - 1 + 1 = e.$$
        So in conclusion, we have that the distribution is $$N - 1 \sim Geom(\dfrac{1}{e})$$ and the expectation is $$E(N) = e.$$
    \item As for the $$min\{n: X_1 + X_2 + ... + X_n \geq 10\}, $$ it is obvious that
        this can be considered as Poisson process as we calculate the sum of $X_i$ and observe them until sum exceeds 10, 
        which is the same as Poisson process is the number of arrivals until the time exceeds 10, where the arrival interval
        follows Expo(1). So we can consider $X_1, X_2, ..., X_n$ as the interarrival times in a Poisson process with rate 1. 
        where the range of the time is $[0, 10)$\\
        Then we have that $$M - 1 \sim Pois(10)$$ and we have that with the property of Poisson distribution, the 
        $E(M)$ is $$E(M) = E(M - 1 + 1) = E(M - 1) + 1 = 10 + 1 = 11$$
        So we have that the distribution is $$M - 1 \sim Pois(10)$$ and the expectation is $$E(M) = 11. $$
    \item As for the $\overline{X}_n$, we have that $$\overline{X}_n = \dfrac{(X_1 + X_2 + ... + X_n)}{n} = \dfrac{X_1}{n} + \dfrac{X_2}{n} + ...  + \dfrac{X_n}{n}$$, 
        As we have that $$X_1, X_2, ... i.i.d. Expo(1),$$ we then have that $$\dfrac{X_1}{n}, \dfrac{X_2}{n}, ... \sim Expo(n)$$
        Then we have that $$\overline{X}_n \sim Gamma(n, n)$$
        As for the approximate distribution of $\overline{X}_n$ for n large, 
        with the center limit theorem, we have that when n is large, the distribution will
        be approximately normal distribution with the same mean and variance as the origin distribution.
        As we have that the origin distribution has mean of 1 and variance of 1, we have that the approximate distribution
        is normal distribution $$\overline{X}_n \sim N(1, \dfrac{1}{n}). $$
        So we have the exact distribution is Gamma distribution $$\overline{X}_n \sim Gamma(n, n)$$ 
        and the approximate distribution is normal distribution $$\overline{X}_n \sim N(1, \dfrac{1}{n}). $$
\end{enumerate}

\end{homeworkProblem}

\newpage

\begin{homeworkProblem}[2]

To show that the inequality 
$$P(|\dfrac{1}{n}\sum_{i = 1}^{n}X_i - \mu| \geq \varepsilon) \leq 2exp(-\dfrac{2n\varepsilon^2}{(b - a)^2}). $$
holds, we use the Hoeffding Lemma + Chernoff Inequality, the Hoeffding Lemma inequality is 
$$E (e^{\lambda x}) \leq e^{\dfrac{1}{8}\lambda ^2(b - a)^2}, $$
the Chernoff Inequality is 
$$P(X \geq a) \leq \dfrac{E(e^{tX})}{e^{ta}}. $$
Proof are as follows:\\


\end{homeworkProblem}

\newpage

\begin{homeworkProblem}[3]
    
\end{homeworkProblem}

\newpage

\begin{homeworkProblem}[4]
    
\end{homeworkProblem}

\newpage

\begin{homeworkProblem}[5]
    
\end{homeworkProblem}

\end{document}
